% ################
% # Setup Pakete #
% ################

% Pakete, die eigene "Setups" haben und ! nicht ! über "Optionen" eingestellt werden können, werden in dieser Datei konfiguriert.


% Farben von Links etc. festlegen
\hypersetup{
%pdfencoding=auto,	% oder unicode
%bookmarks=true, 	% ist Standard
unicode=true,
bookmarksopen=true,
bookmarksnumbered=true,
pdftitle={\varTitelHyper},
pdfauthor={\varAutorEins},
pdfsubject={Bachelor-Thesis},
pdfcreator={Texpad und Mac OS X},
pdfkeywords={\varKeyA}{\varKeyB}{\varKeyC}{\varKeyD}{\varKeyE},
colorlinks=true,
linkcolor=green,
citecolor=RawSienna,
filecolor=magenta,
urlcolor=cyan
}
% #####-------------------------------------------------#####

% #####-------------------------------------------------#####
% Einstellungen für Chemie-Pakete "circled=formal":
% \mch		normales Minus-Zeichen 
% \pch		normales Plus-Zeichen
% \fmch		umkreistes Minus-Zeichen
% \fpch		umkreistes Plus-Zeichen
% \el		Elektron mit normalem Minus 
% \prt		Proton mit normalem Plus
\chemsetup{language=english,
modules=all, 
%formula = chemformula
}
\chemsetup[phases]{pos=sub} 
% Nummerierung der Reaktions-Formeln nach Chapter (R 1.1)
\usechemmodule{reactions}
\renewtagform{reaction}[R ]{(}{)}















