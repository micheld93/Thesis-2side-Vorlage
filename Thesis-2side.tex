% ##################################################
% #                                                #
% # Autor: B. Michel Döhring                       #
% # E-Mail: micheld.93@gmail.com                   #
% # Version: 2.0                                   #
% # Datum: Oktober 2017                            #
% #                                                #
% ##################################################


% ######################################
% # Konfigurieren der Dokumentenklasse #
% ######################################

\documentclass[
    a4paper,				% Papierformat
    abstract=on,		   	% Abstract mit Überschrift
    openright,				% Start Chapter rechte Seite
    numbers=noenddot,		% kein Punkt nach Kapitelnummer
    parskip=half	,		% keine Absatzeinzüge, Sprung zwischen Absätzen
    twoside,            		% Zweiseitig
    12pt,               		% Schriftgröße
    pagesize=auto,      		% schreibt die Papiergröße korrekt ins Ausgabedokument
    headsepline=on,  		% Linie unter der Kopfzeile
    captions=tableheading,
    listof=totoc,   		% Verzeichnisse im Inhaltsverzeichnis
    bibliography=totoc,
    index=totoc 
]{scrreprt}


% ####################
% # Pakete einbinden #
% ####################
% Pakete erweitern LaTeX um zusätzliche Funktionen.
% Weitere sollten in der Datei <01_EigenePakete.tex> hinzugefügt werden.
\usepackage{scrhack}
% Legt die Zeichenkodierung fest, z.B UTF8
\usepackage[utf8]{inputenc}
% Verwendung der Zeichentabelle T1, für deutschsprachige Dokumente sinnvoll
\usepackage[T1]{fontenc}
% Silbentrennung nach neuer deutscher und englischer Rechtschreibung
\usepackage[english,french,ngerman]{babel}
% Mehr Farben als den Standard
\usepackage[usenames,dvipsnames,table,xcdraw]{xcolor}
% #####-------------------------------------------------#####

% #####-------------------------------------------------#####
% #########################
% # Mathe und NaWi Pakete #
% #########################
% Mathepaket für align-Umgebung
\usepackage{amsmath}
% Mathepaket für erweiterungen von "amsmath"
\usepackage{mathtools}
% Mathe und sonstige Symbole
% "amsfonts" wird automatisch geladen, msam, msbm, eufm Schriften
% \usepackage{amsfonts}
\usepackage{amssymb}
% Ermöglicht die Nutzung von \SI[per-mode=fraction]{1,345}{\coulomb\per\mole}
\usepackage[
	locale=DE,
	per-mode=fraction,
	output-decimal-marker={{,}},
	separate-uncertainty,
	range-phrase={{ bis }},
]{siunitx}
%Für Origin-Symbole:
\usepackage{oplotsymbl}
% Chemie-Pakete
% chemmacros
% hat eigenes Setup in 03_SetupPakete
\usepackage{chemmacros}
%\usepackage{chemfig}
% #####-------------------------------------------------#####

% #####-------------------------------------------------#####
% ##################################
% # Tabellen- und Abbildungspakete #
% ##################################
% Für Gleitposition H
\usepackage{float}
% Multirow in Tabellen:
\usepackage{multirow}
% Für schöne Tabellen, bietet dicke/dünne Linien an
\usepackage{booktabs}
\renewcommand{\arraystretch}{1.3}
\setlength{\tabcolsep}{5pt}
% #####-------------------------------------------------#####
% #####-------------------------------------------------#####
% Beide Pakete werden für die Ausrichtung der Tabellenspalten benötigt
\usepackage{array}
% Für lange Tabellen
\usepackage{longtable}
% Zum flexiblen Einbinden von Grafiken, [pdftex] ist Option
\usepackage{graphicx}
%Zum Einbinden von PDFs im Dokument
%\usepackage{pdfpages}
% #####-------------------------------------------------#####
% #####-------------------------------------------------#####
% Darstellung für Caption
% hat eigenes Setup in 03_SetupPakete
\usepackage[font=small,labelfont=bf,labelsep=endash,format=plain]{caption}
% Fließtext um Figure-Umgebung
\usepackage{wrapfig}
% #####-------------------------------------------------#####

% #####-------------------------------------------------#####
% ##########################
% # Schriften, Listen etc. #
% ##########################
% Einfaches wechseln zwischen unterschiedlichen Zeilenabständen
\usepackage{setspace}
% Zusatzfunktionen zum zitieren
\usepackage{cite}
% Für Notizen:
\usepackage{todonotes}
% Wird für Kopf- und Fußzeile benötigt
\usepackage[automark]{scrlayer-scrpage}
\automark [chapter]{chapter}
\automark*[section]{}
% Andere Nummerierstile bei enumerate
\usepackage{paralist}
% Index-Verzeichnis
\usepackage{makeidx}
% #####-------------------------------------------------#####

% #####-------------------------------------------------#####
% #############################################
% # Konfiguration des Abkürzungsverzeichnises #
% #############################################
% ACRO-Paket einbinden
\usepackage{acro}
% ACRO einstellungen
\acsetup{			
	only-used=false,			% auch die nicht verwendeten in die Liste aufnehmen
	hyperref=true,		    % Link der Acros im Text
	sort=true,  				% Liste sortieren
    list-heading=addchap
}
% #####-------------------------------------------------#####

% #####-------------------------------------------------#####
% ############################
% # weitere Pakete einbinden #
% ############################
% #################
% # Eigene Pakete #
% #################

%====================================================================================%         
%====================Ab hier kommen die neuen zusätzlichen Pakete====================%
%====================================================================================%








% #####-------------------------------------------------#####

% #####-------------------------------------------------#####
% ###############################
% # Randeinstellungen festlegen #
% ###############################
% Rand-Einstellungen für das Dokument
\usepackage{geometry}
\geometry{verbose,nomarginpar,lmargin=35mm,rmargin=20mm,bmargin=30mm}
% #####-------------------------------------------------#####

% #####-------------------------------------------------#####
% ####################################
% # Grund-Konfiguration von Hyperref #
% ####################################
% Verlinkt Textstellen im PDF Dokument
% erst nach allen Paketen laden
% aber vor "geometry" laden
% hat eigenes Haupt-Setup in 03_SetupPakete
\usepackage[pdfpagelabels,
plainpages=false,
bookmarksopen]{hyperref}
% #####-------------------------------------------------#####

% #########################
% # Variablen importieren #
% #########################

% ####################
% # Eigene Variablen #
% ####################

% Der Befehl \newcommand kann auch benutzt werden um Variablen zu definieren:

% Titel der Arbeit:
    \newcommand{\varTitel}{Titel der Arbeit}

% Titel der Arbeit (Hyper-Kompatibel):
    \newcommand{\varTitelHyper}{Hyperref-komp. Name}

% Englischer Titel der Arbeit:
\newcommand{\varTitelEng}{English Title}

% Zur Erlangung eines Akademischen Grades:
    \newcommand{\varAkademGrad}{Zur Erlangung des akademischen Grades\\
    	Master of Science (M.Sc.)\\
    	vorgelegte Thesis}

% Datum:
    \newcommand{\varDate}{2017}

% Abgabetag oder -Datum für die Thesis:
    \newcommand{\varDateII}{XX.YY.2017}
        
% Autoren der Thesis:
    \newcommand{\varAutorEins}{Autor} 

% Ort:
\newcommand{\varOrt}{Ort}

% Erster Betreuer der Thesis:
    \newcommand{\varBetreuer}{Prof. Dr. ??} 
    
% Zweiter Betreuer der Thesis:
    \newcommand{\varZweitgutachter}{Prof. Dr. ??} 
    
% E-Mail-Adressen der Autoren:
    \newcommand{\varEmail}{Mail-Adresse}

% Fachbereich:
    \newcommand{\varFachbereich}{Fachbereich XX: Physik, Mathematik...}

% Institut:
    \newcommand{\varInstitut}{
    Betreuendes Institut der Arbeit
    }

% Schlagworte und Keywords:
    \newcommand{\varKeyA}{Master}
    \newcommand{\varKeyB}{Thesis}
    \newcommand{\varKeyC}{Msc}
    \newcommand{\varKeyD}{Messtechnik}
    \newcommand{\varKeyE}{}
    
% Stil der Einträge im Literaturverzeichnis
    %\newcommand{\varLiteraturLayout}{unsrtdinetal}
     \newcommand{\varLiteraturLayout}{unsrtdinetalDOI}


% ################################
% # Pakete Einstellungen / Setup #
% ################################

% ################
% # Setup Pakete #
% ################

% Pakete, die eigene "Setups" haben und ! nicht ! über "Optionen" eingestellt werden können, werden in dieser Datei konfiguriert.


% Farben von Links etc. festlegen
\hypersetup{
%pdfencoding=auto,	% oder unicode
%bookmarks=true, 	% ist Standard
unicode=true,
bookmarksopen=true,
bookmarksnumbered=true,
pdftitle={\varTitelHyper},
pdfauthor={\varAutorEins},
pdfsubject={Bachelor-Thesis},
pdfcreator={Texpad und Mac OS X},
pdfkeywords={\varKeyA}{\varKeyB}{\varKeyC}{\varKeyD}{\varKeyE},
colorlinks=true,
linkcolor=green,
citecolor=RawSienna,
filecolor=magenta,
urlcolor=cyan
}
% #####-------------------------------------------------#####

% #####-------------------------------------------------#####
% Einstellungen für Chemie-Pakete "circled=formal":
% \mch		normales Minus-Zeichen 
% \pch		normales Plus-Zeichen
% \fmch		umkreistes Minus-Zeichen
% \fpch		umkreistes Plus-Zeichen
% \el		Elektron mit normalem Minus 
% \prt		Proton mit normalem Plus
\chemsetup{language=english,
modules=all, 
%formula = chemformula
}
\chemsetup[phases]{pos=sub} 
% Nummerierung der Reaktions-Formeln nach Chapter (R 1.1)
\usechemmodule{reactions}
\renewtagform{reaction}[R ]{(}{)}

















% ############################
% # Eigene Befehle einbinden #
% ############################

% ##################
% # Eigene Befehle #
% ##################
% Eigene Befehle eignen sich gut um Abkürzungen für lange Befehle zu erstellen. Die Syntax ist folgende:

% \newcommand{neuer Befahl}{ein langer Befehl}




% ##############################
% # Eigene Einheiten einbinden #
% ##############################

% ####################
% # Eigene Einheiten #
% ####################

% Eigene Einheiten, definiert mit dem SIUNITX-Paket kommen hier herein!
% \DeclareSIUnit\degree{}



% ###########################
% # Konfiguration des Index #
% ###########################
\makeindex
% Worte werden durch \index{C!Crossed-beams|textbf} ins Verzeichnis aufgenommen

% #####################################
% # Abkürzungsverzeichnis importieren #
% #####################################

% ###########################################################
% # Abkürzungsverzeichnis und weitere Acronym-Verzeichnisse #
% ###########################################################




% #########################
% # Abkürzungsverzeichnis #
% #########################

\DeclareAcronym{IAMP}{
  short = IAMP,
  long  = Institut für Atom- und Molekülphysik,
}
\DeclareAcronym{AGAMOP}{
  short = AMoP,
  long  = Arbeitsgruppe für Atom- und Molekülphysik,
}
\DeclareAcronym{CAD}{
  short = CAD,
  long  = Computer Aided Design,
  extra = ,
}



% #####################
% # Symbolverzeichnis #
% #####################

\DeclareAcronym{elektron}{
  short = $\el$ ,
  long  = Elektron ,
  extra =  ,
  class = symbole
}
\DeclareAcronym{eladung}{
  short = $e$ ,
  long  = Elementarladung ,
  extra = $$ ,
  class = symbole
}


% ###########################
% # Weitere Konfigurationen #
% ###########################

% Serifen für Überschriften
\addtokomafont{sectioning}{\rmfamily}
% Nummerierung der Formeln entsprechend der Chapter (z.B. 1.1)
\numberwithin{equation}{chapter}
% Ändert Schriftgröße und Zeilenabstand bei captions
\addtokomafont{caption}{\small\linespread{1}\selectfont}

% #######################################
% # Kopf- und Fußzeile konfigurieren    #
% #######################################

\clearpairofpagestyles
\ihead{}									% Innenseite der Kopfzeile
\chead{}                                    % Mitte der Kopfzeile
\ohead{\headmark}              				% Außenseite der Kopfzeile
\ifoot{}                                    % Innnenseite der Fußzeile
\cfoot[- \pagemark{} -]{- \pagemark{} -}  	% Mitte der Fußzeile
\ofoot{}    								% Aussenseite der Fußzeile

\addtokomafont{pagenumber}{\normalfont}
\pagestyle{scrheadings}

% ####################
% # Beginn Doukument #
% ####################

\begin{document}

% 1.5er Zeilenabstand
\onehalfspacing
% Römische Ziffern als Seitenzahlen für Titelseite bis einschließlich dem Inhaltsverzeichnis
\pagenumbering{roman}

% ########################
% # Titelseite einbinden #
% ########################

\begin{titlepage}
    \begin{center}
        \varAkademGrad \\
        \vspace{0.8cm}
        \Large{
        \textbf{\varTitel}
        }
        \\
        \vspace{0.6cm}
        \large{
        \textit{\varTitelEng}
        }
        \\
        \vspace{2.0cm}

        
        \Large{
        \textbf{\varAutorEins} \\
        }
        \large{
        \varOrt
        } 
       
        \vspace{0.5cm}
        
        
        \normalsize{
        \varDate \\
        \vspace{1.0cm}
        }       
        
        \normalsize
		\begin{table}[H]
			\centering
			\begin{tabular}{ll}
  			  Erstgutachter: & \varBetreuer \\ 
 		      Zweitgutachter: & \varZweitgutachter \\ 
			\end{tabular}
		\end{table} 

		\vspace{1.0cm}
        

        \begin{figure}[H]
			\centering
			\includegraphics[width=7cm]{example-image}
			% \includegraphics[width=8cm]{Bilder/IAMP}
		\end{figure}

        \textbf{\varFachbereich} \\
        \vspace{0.5cm}
        \textbf{\varInstitut}
        
		%\textit{\varEmail}\\
    \end{center}
\end{titlepage}




% ################################
% # Inhaltsverzeichnis einbinden #
% ################################
% Farbe der Links auf schwarz
\hypersetup{linkcolor=black}
\tableofcontents

% ########################################
% # Anzeigen des Abkürzungsverzeichnises #
% ########################################

%====================Beginn des Abkürzungsverzeichnisses====================%

% alle außer der Klasse 'symbole' auflisten:
\acsetup{
	list-style  = extra-longtable,
	extra-style = plain,
	list-short-format = {\bfseries}
}

\printacronyms[exclude-classes=symbole, name=Abkürzungsverzeichnis]

%====================Beginn des Symbolverzeichnisses====================%

% nur die Klasse 'symbole' auflisten:
\acsetup{
  list-style  = extra-longtable-rev,
  extra-style = plain
}

\printacronyms[include-classes=symbole, name=Symbolverzeichnis]

% #####################################
% # Das Abstract der Arbeit einbinden #
% #####################################

\cleardoublepage
\markboth{}{} % Abstract ändert die Kopfzeile nicht

\pagestyle{scrplain}

% ##################################
% # Abstract und Zusammenfassungen #
% ##################################


\begin{otherlanguage}{ngerman} 
	\begin{abstract} 
		Deutsch
	\end{abstract} 
\end{otherlanguage}


%\cleardoublepage
% englische Zusammenfassung

\begin{otherlanguage}{english} 
	\begin{abstract} 
		Englsich  
	\end{abstract}  
\end{otherlanguage}




% französische Zusammenfassung

\begin{otherlanguage}{french} 
	\begin{abstract} 
		Französisch	
	\end{abstract} 
\end{otherlanguage} 

% ainsi: so, also, daher, folglich
% celle-ci: diese
% l'ancien: alt


\pagestyle{scrheadings}

%=======================================================================%
%=======================================================================%


% Ab hier mit Kopf- und Fußzeile
% ###################################
% # Den Inhalt der Arbeit einbinden #
% ###################################
\clearpage
\pagenumbering{arabic}

% Farbe der Links auf rot
\hypersetup{linkcolor=red}
% Kein Gleit-Absatz bei twoside-Format
\raggedbottom

% ##########################
% # Eigene Wort-Trennungen #
% ##########################


\hyphenation{ChenYang}

\chapter{Einleitung}
\input{301_Inhalt}
\input{302_Inhalt}
\input{303_Inhalt}
\input{304_Inhalt}
\input{305_Inhalt}
\input{306_Inhalt}
\input{307_Inhalt}
\input{308_Inhalt}
\input{309_Inhalt}
  
% ####################
% # Anhang einbinden #
% ####################
\appendix
	% Maximal eine "\chapter"-Umgebung angeben
% Für Vorder-Rückseite unbedingt ab "Anhang" kontrollieren!

\chapter{Anhang-A}\label{kap:anhang-A0}

In diesem Kapitel sind alle im Hauptteil erwähnten CAD-Zeichnungen zu finden.






		\newpage
 	    	% Maximal eine "\chapter"-Umgebung angeben
% Für Vorder-Rückseite unbedingt ab "Anhang" kontrollieren!

\section{Anhang-A1}\label{kap:anhang-A1}
%Für hyperref:  \texorpdfstring{Angezeigter Text in PDF}{HyperRef kompatibel}












        \newpage
    		% Maximal eine "\chapter"-Umgebung angeben

\section{Anhang-A2}\label{kap:anhang-A2}





        %\newpage
    		%\input{403_Anhang-A}
		\newpage
    % Maximal eine "\chapter"-Umgebung angeben

\chapter{Anhang-B}\label{kap:anhang-B0}


		\newpage
 	   		% Maximal eine "\chapter"-Umgebung angeben

\section{Anhang-B1}\label{kap:anhang-B1}


        \newpage
    		% Maximal eine "\chapter"-Umgebung angeben
% Für Vorder-Rückseite unbedingt ab "Anhang" kontrollieren!

\section{Anhang-B2}\label{kap:anhang-B2}


        \newpage
    		% Maximal eine "\chapter"-Umgebung angeben
% Für Vorder-Rückseite unbedingt ab "Anhang" kontrollieren!

\section{Anhang-B3}\label{kap:anhang-B3}






   	    \newpage
        	% Maximal eine "\chapter"-Umgebung angeben

\section{Anhang-B4}\label{kap:anhang-B4}

   	    \newpage
        % Maximal eine "\chapter"-Umgebung angeben
% \texorpdfstring{pTaxt}{Text}

\section{Anhang-B5}\label{kap:anhang-B5}


%		\newpage
%	\input{600_Anhang-C}
%		\newpage
%   		\input{601_Anhang-C}
%  		\newpage
%			\input{602_Anhang-C}
%    	\newpage
%			\input{603_Anhang-C}
% 	\input{700_Anhang-D}
%		\newpage
%    		\input{701_Anhang-D}
%    	\newpage
%			\input{702_Anhang-D}
%   	\newpage
%			\input{703_Anhang-D}    	

% ###################################
% # Literaturverzeichnis mit BibTeX #
% ###################################
    
\bibliography{literatur}
\bibliographystyle{\varLiteraturLayout}

% #######################################
% # Abbildungs- und Tabellenverzeichnis #
% #######################################

% Farbe der Links auf schwarz
\hypersetup{linkcolor=black}

%Abbildungsverzeichnis
\listoffigures

%Tabellenverzeichnis
\listoftables

% ##############
% # Danksagung #
% ##############

% ##############
% # Danksagung #
% ##############


\addchap{Danksagung} % Chapter ohne Nummer

Danke


% ##################################
% # Selbstständigkeitsversicherung #
% ##################################

% ##################################
% # Selbstständigkeitsversicherung #
% ##################################


\addchap{Selbstständigkeitserklärung} % Chapter ohne Nummer

Irgendeine entsprechende Erklärung


\vspace{2.0cm}


\begin{minipage}{0.49\linewidth}
	\begin{tabular}{lcr}
		\hline
		\hspace{1.5cm} & Ort, Datum & \hspace{1.5cm}
	\end{tabular}
\end{minipage}
		\hfill
\begin{minipage}{0.4\linewidth}
	\begin{tabular}{lcr}
		\hline
		\hspace{1.5cm} & Unterschrift & \hspace{1.5cm}
	\end{tabular}		
\end{minipage}


% ########################
% # Index des Dokumentes #
% ########################

% Farbe der Links auf rot
\hypersetup{linkcolor=red}

\printindex

% #######################
% # Ende des Dokumentes #
% #######################

\end{document}
